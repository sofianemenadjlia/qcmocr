%%
%% Examen de PdP
%%
%% Copyright (C) 2014-2024, LaBRI, Université de Bordeaux.
%%
%% History:
%% - 26 février 2014: Début du template.
%% - 16 mars    2015: Version 2015
%% - 4 janvier  2024: Template générique pour les Quiz de PdP

\documentclass[a4paper]{article}

%% Font packages
\usepackage{lmodern}
\usepackage[T1]{fontenc}
\usepackage[utf8]{inputenc}

%% Localization
\usepackage[french]{babel}

%% Other packages
\usepackage{multicol}

%% AMC package for multiple-choice questions
\usepackage[francais,bloc,completemulti]{automultiplechoice}

%% The same but with the answers
%\usepackage[francais,bloc,completemulti,answers]{automultiplechoice}

%% Page layout
\addtolength{\textheight}{2em}
\addtolength{\textwidth}{2cm}
\addtolength{\hoffset}{-1cm}

%% Document
\begin{document}

%% Scoring strategies
%% Parameters (this idiot chose french names for the parameters...):
%%   * e (erreur) : réponse incohérente (plusieurs cases cochées pour une
%%       question à choix unique ou pour, une question à choix multiple, la
%%       réponse 'aucune réponse' cochée avec une autre réponse)
%%   * v (vide) : aucune case cochée
%%   * b (bonne) : bonne réponse
%%   * m (mauvaise) : mauvaise réponse
%%   * p (plancher) : nombre de points minimum pour la question
%%   * mz (maximum ou zéro) : toutes les réponses correctes ou zéro point

% Single choice question
\scoringDefaultS{e=-1,v=0,b=1,m=-1}

% Multiple choices questions
\scoringDefaultM{e=-1,v=0,b=1,m=-1,p=-1,formula=((NBC-NMC>2)?2:(NBC-NMC))}
%\scoringDefaultM{e=-1,v=0,b=1,m=-1,p=-1,formula=((NBC-NMC>2)?2:(NBC-NMC))}

%% Questions
\input{questions.tex}

%% Exam copy (with the number of copies)
\begin{examcopy}[125]

  %% Header for the exam
  \begin{minipage}{.4\linewidth}
    \centering\large\bf Projets de Progammation\\ Quiz (\examtopic)
  \end{minipage}
\noindent\fbox{%
\namefield{\begin{minipage}{.55\linewidth}
    \begin{minipage}[t]{.55\linewidth} % Adjust the width as needed
        Nom, Prénom:
        
        \vspace*{.5cm}\dotfill
        \vspace*{1mm}
    \end{minipage}%
    \vrule % Vertical line
    \begin{minipage}[t]{.35\linewidth} % Adjust the width as needed
          Num étudiant:
        
        \vspace*{.5cm}\dotfill
        \vspace*{1mm}
    \end{minipage}
\end{minipage}}%
}

  %% End of the header

  %% Rules and warnings for the exam
  \medskip
  %
  \begin{minipage}{\textwidth}
    \textbf{Indications et règles:}

    \begin{itemize}
      \item Aucun document n'est autorisé.

      \item Remplir/cocher les cases de manière nette et précise (aucune rature ou annotation supplémentaire n'est possible, car ce QCM sera corrigé par reconnaissance optique automatique).

      \item Les questions simples requièrent de cocher une seule case, et valent +1 pour une bonne réponse, -1 pour une mauvaise réponse, et 0 si aucune case n'est cochée.

      \item Les questions avec un '\multiSymbole{}' requièrent de cocher de une à plusieurs cases. Ces questions valent +1 par case correcte cochée, -1 par case incorrecte cochée, et 0 dans les autres cas (néanmoins, la valeur totale sera bornée inférieurement par~$-1$ et supérieurement par~$+2$).
    \end{itemize}
  \end{minipage}
  %

  %% Header for the question list
  \begin{center}
    \hrule\vspace{2mm}
    \bf\Large Questions
    \vspace{1mm}
    \hrule
    \bigskip
  \end{center}

  %% Question list
  \shufflegroup{main}
  %% if you want to show only 10 questions from the group (taken randomly)
  %% \insertgroup[10]{main}
  \insertgroup[10]{main}

  \clearpage

\end{examcopy}

\end{document}
