%
% Copyright (C) 2012-2021 Alexis Bienvenüe
%
% This file is part of Auto-Multiple-Choice
%
% Auto-Multiple-Choice is free software: you can redistribute it
% and/or modify it under the terms of the GNU General Public License
% as published by the Free Software Foundation, either version 2 of
% the License, or (at your option) any later version.
%
% Auto-Multiple-Choice is distributed in the hope that it will be
% useful, but WITHOUT ANY WARRANTY; without even the implied warranty
% of MERCHANTABILITY or FITNESS FOR A PARTICULAR PURPOSE.  See the GNU
% General Public License for more details.
%
% You should have received a copy of the GNU General Public License
% along with Auto-Multiple-Choice.  If not, see
% <http://www.gnu.org/licenses/>.
%
\documentclass[a4paper]{article}

%T[PACKAGES]
%<PACKAGES
\usepackage[utf8x]{inputenc}
\usepackage[T1]{fontenc}

\usepackage{fp}% http://tug.ctan.org/tex-archive/macros/latex/contrib/fp/README
\usepackage{multicol}
\usepackage{amssymb}
\usepackage[right]{eurosym}

\usepackage[frenchb]{babel}
%>PACKAGES

\usepackage[bloc,completemulti,calibration,correcindiv]{automultiplechoice}

\begin{document}

%%%%%%%%%%%%%%%%%%%%%%%%%%%%%%%%%%%%%%%%%%%%%%%%%%%%%%%%%%%%%%%%%%%%%%
%%%%%%%%%%%%%%%%%%%%%%%%%%%%%%%%%%%%%%%%%%%%%%%%%%%%%%%%%%%%%%%%%%%%%%

%T[PREAMBULE]
%<PREAMBULE

\def\implique{Quelles affirmations peut-on déduire de ces seules hypothèses ?}
\def\indique#1{}

\def\retaille{\displaystyle\vphantom{\vbox to 6ex{}}}

\FPseed=100

\def\deuxacinq{
\FPeval\gogol{trunc(2 + random * 4,0)}
\gogol}

\def\choixProba#1{
    \begin{multicols}{5}
      \begin{reponses}[o]
        \choixIntervalles{#1}{0}{1}{0.05}
      \end{reponses}
    \end{multicols}
}  
\def\CC{\mathbb{C}}
\def\complexe{\mathbb{C}}
\def\RR{\mathbb{R}}
\def\real{\mathbb{R}}
\def\integer{\mathbb{N}}
\def\NN{\mathbb{N}}  
\def\ZZ{\mathbb{Z}}   
\def\relative{\mathbb{Z}}   
\def\QQ{\mathbb{Q}} 
\def\rationel{\mathbb{Q}}
\def\II{\mbox{\bf 1}}
\def\PP{\mathbb{P}}
\def\QQ{\mathbb{Q}}
\def\FF{\mathbb{F}}

%%%%% Lettres caligraphiees %%%%%%%%
\def\Fcal{{\cal{F}}}
\def\Exp{{\cal{E}}}
\def\Gcal{{\cal{G}}}
\def\Geom{{\cal{G}}}
\def\Cont{{\cal{C}}}
\def\Cauchy{{\cal{C}}}
\def\Multin{{\cal{M}}}
\def\Norm{{\cal{N}}}
\def\Acal{{\cal{A}}}
\def\tribu{{\cal{A}}}
\def\Bcal{{\cal{B}}}
\def\Binom{{\cal{B}}}
\def\Borel{{\cal{B}}}
\def\Ccal{{\cal{C}}}
\def\Ecal{{\cal{E}}}
\def\Ocal{{\cal{O}}}
\def\Mcal{{\cal{M}}}
\def\Unif{{\cal{U}}}
\def\Xcal{{\cal{X}}}
\def\Ycal{{\cal{Y}}}
\def\Lcal{{\cal{L}}}
\def\Pcal{{\cal{P}}}
\def\Scal{{\cal{S}}}
\def\Tcal{{\cal{T}}}

% opérateurs mathématiques
\def\Esp{\mathop{\rm E\vphantom{s}}\nolimits}
\def\Proba{\mathop{\PP{}}\nolimits}

\def\preS{\,}
\def\postS{}
\renewcommand{\P}[2][]{\Proba^{#1}\left[ #2 \right]}
\newcommand{\Ppar}[2][]{\Proba^{#1}\left( #2 \right)}
\def\Psachant#1#2{\Proba\left[\left. #1\vphantom{#2} \preS\right|\postS #2 \right]}
\def\PsachantP#1#2{\Proba\left(\left. #1\vphantom{#2} \preS\right|\postS #2 \right)}
\def\PsachantPS#1#2{\Proba( #1 \preS|\postS #2 )}
\newcommand{\Esachant}[3][]{\Esp^{#1}\left( #2\vphantom{#3} \preS\right|\postS \left. #3\vphantom{#2} \right)}
\newcommand{\Varsachant}[2]{\Var\left( #1\vphantom{#2} \preS\right|\postS \left. #2\vphantom{#1} \right)}
\def\EsachantS#1#2{\Esp( #1 \preS|\postS #2 )}
\def\EsachantST#1#2{\Esp\left( #1 \preS|\postS #2 \right)}
\def\sachant{\mid}

\def\indic#1{{\rm\bf 1}_{\left\{#1\right\}}}
\def\indice#1{{\rm\bf 1}_{#1}}
\def\indich#1{{\rm\bf 1}\left\{#1\right\}}
\def\Cov{\mathop{\rm Cov}\nolimits}
\def\Var{\mathop{\rm Var}\nolimits}
\def\Card{\mathop{\rm Card}\nolimits}
\def\Trace{\mathop{\rm Trace}\nolimits}
\def\logit{\mathop{\rm logit}\nolimits}
\def\signe{\mathop{\rm signe}\nolimits}

\def\pscal#1#2{\langle #1,#2\rangle}
\def\norme#1{\left\| #1 \right\|}
\def\pent#1{\left\lfloor #1\right\rfloor}

\def\trace{\mathop{\rm tr}}
\def\transp#1{\,\vphantom{#1}^t#1}
\def\Pt{\,^t\!P}
\def\At{\,^t\!\!A}  

%>PREAMBULE

\AMCinterIrep=.75ex
\AMCrandomseed{1237893}

\FPmessagesfalse

\setlength{\multicolsep}{.7ex}


%%%%%%%%%%%%%%%%%%%%%%%%%%%%%%%%%%%%%%%%%%%%%%%%%%%%%%%%%%%%%%%%%%%%%%
%%%%%%%%%%%%%%%%%%%%%%%%%%%%%%%%%%%%%%%%%%%%%%%%%%%%%%%%%%%%%%%%%%%%%%

%Q[Calc-bande-unif]
\element{grq}{
\begin{question}{Calc-bande-unif}\bareme{b=3}
  \FPeval\VQk{trunc(3+4*random,0)}
  \FPeval\VQr{round(1-(1-1/VQk)^2,10)}
  Soient $X$ et $Y$ deux variables aléatoires indépendantes toutes
  deux de loi uniforme sur $[0,1]$.
  
  À quel intervalle appartient $\Proba(|X-Y|\le1/\VQk)$ ? \indique{\VQr}

  \choixProba{\VQr}
\end{question}
}

%Q[Calc-restric]
\element{grq}{
\begin{question}{Calc-restric}\bareme{b=3}
  \FPeval\VQn{trunc(3+2*random,0)}
  \FPeval\VQi{trunc(1+2*random,0)}
  \FPeval\VQj{round(VQi+1,0)}
  \FPeval\VQr{round( (VQj^4-VQi^4)/(VQn^3)*3/4 ,10)}
  Soit $X$ une variable aléatoire réelle de densité
  \[ f(x)=\alpha x^2 \II_{[0,\VQn]}(x)\,,\]
  où $\alpha$ est une constante.

  À quel intervalle appartient $\Esp(X\II_{[\VQi,\VQj]}(X))$ ? \indique{\VQr}
  
  \begin{multicols}{5}
    \begin{reponses}[o]
      \choixIntervalles{\VQr}{0}{2}{0.1}
    \end{reponses}
  \end{multicols}
\end{question}
}

%Q[Calc-frep-d2]
\element{grq}{
\begin{question}{Calc-frep-d2}\bareme{b=3}
  \FPeval\VQa{round(trunc(2+7*random,0)/10,1)}
  \FPeval\VQb{round(trunc(2+7*random,0)/10,1)}
  \FPeval\VQr{round( (VQa*VQb-ln(1+VQa*VQb))/VQa/(1-ln(2)) ,10)}
  On considère un vecteur aléatoire $(X,Y)$ de densité
  \[ f(x,y)=\alpha\frac{y}{(1+xy)^2}\II_{[0,1]²}(x,y)\,,\]
  où $\alpha$ est une constante.

  À quel intervalle appartient $\Proba(X<\VQa\mbox{~et~}Y<\VQb)$ ? \indique{\VQr}

  \choixProba{\VQr}
\end{question}
}

%Q[Calc-nocalc]
\element{grq}{
\begin{question}{Calc-nocalc}\bareme{b=2}
  \FPeval\VQn{trunc(1+8*random,0)}
  \FPeval\VQnc{round(VQn+1,0)}
  \FPeval\VQnn{round(VQn+2,0)}
  \FPeval\VQh{trunc(1+8*random,0)}
  \FPeval\VQhh{trunc(VQh+1+4*random,0)}
  On considère un vecteur aléatoire $(X,Y)$ de densité
  \[ f(x,y)=\alpha \II_{[\VQn,\VQnn]}(x)\II_{[\VQh,\VQhh]}(y)\frac{x+y}{100-\ln(xy/100)}\,,\]
  où $\alpha$ est une constante.

  À quel intervalle appartient $\Esp(X)$ ? \indique{\VQnc}

  \begin{multicols}{5}
    \begin{reponses}[o]
      \choixIntervalles{\VQnc}{0}{10}{1}
    \end{reponses}
  \end{multicols}
\end{question}
}

%Q[Calc-cov-xu-x]
\element{grq}{
\begin{question}{Calc-cov-xu-x}\bareme{b=3}
  \FPeval\VQe{trunc(1+8*random,0)}
  \FPeval\VQv{round(2*trunc(1+4*random,0),0)}
  \FPeval\VQvxu{round( (VQv+(VQv+VQe^2)/3)/4 ,10)}
  \FPeval\VQr{round( VQv/2/((VQvxu*VQv)^(1/2)) ,10)}
  On considère deux variables aléatoires $X$ et $U$ indépendantes. On
  suppose que $U$ suit la loi uniforme sur $[0,1]$, que $\Esp(X)=\VQe$
  et que $\Var(X)=\VQv$.

  À quel intervalle appartient le coefficient de corrélation linéaire de $X$ et $XU$ ? \indique{\VQvxu -- \VQr}

  \begin{multicols}{5}
    \begin{reponses}[o]
      \choixIntervalles{\VQr}{-1}{1}{0.1}
    \end{reponses}
  \end{multicols}
\end{question}
}

%Q[Calc-comp-rapport]
\element{grq}{
\begin{question}{Calc-comp-rapport}\bareme{b=3}
  \FPeval\VQa{trunc(2+3*random,0)}
  \FPeval\VQb{trunc(VQa+1+3*random,0)}
  \FPeval\VQr{round( (VQa/VQb)^3 ,10)}
  Soit $(X,Y)$ un vecteur aléatoire de densité
  \[ f(x,y)=\alpha \left(\frac yx \right)^2 \II_{]0,1]}(x)\II_{[0,\VQb x]}(y)\,,\]
  où $\alpha$ est une constante.

  À quel intervalle appartient $\Proba(Y<\VQa X)$ ? \indique{\VQr}

  \choixProba{\VQr}
\end{question}
}

%Q[Calc-inv-poiss]
\element{grq}{
\begin{question}{Calc-inv-poiss}\bareme{b=3}
  \FPeval\VQl{trunc(2+4*random,0)}
  \FPeval\VQr{round( (1-exp(-VQl))/VQl ,10)}
  Soit $X$ une variable aléatoire de loi de Poisson de paramètre \VQl.
  À quel intervalle appartient $\Esp(1/(1+X))$ ? \indique{\VQr}

  \choixProba{\VQr}
\end{question}
}

%Q[Calc-convol]
\element{grq}{
\begin{question}{Calc-convol}\bareme{b=3}
  \FPeval\VQl{trunc(2+2*random,0)}
  \FPeval\VQz{trunc(1+3*random,0)}
  \FPeval\VQa{trunc(VQz+1+2*random,0)}
  \FPeval\VQr{round( (1-exp(-VQz/VQl))/VQa ,10)}
  Soit $X$ une variable aléatoire de loi exponentielle de paramètre
  $1/\VQl$ et $U$ une variable aléatoire de loi uniforme sur
  $[0,\VQa]$. On suppose que $X$ et $U$ sont indépendantes. On note $f$ la densité de la variable $(X+U)$. 

  À quel intervalle appartient $f(\VQz)$ ?  \indique{\VQr}

  \begin{multicols}{5}
    \begin{reponses}[o]
      \choixIntervalles{\VQr}{0}{0.5}{0.025}
    \end{reponses}
  \end{multicols}
\end{question}
}

%Q[Calc-changt-var]
\element{grq}{
\begin{question}{Calc-changt-var}\bareme{b=3}
  \FPeval\VQl{trunc(2+2*random,0)}
  \FPeval\VQll{trunc(2+2*random,0)}
  \FPeval\VQu{trunc(1+2*random,0)}
  \FPeval\VQv{trunc(1+2*random,0)}
  \FPeval\VQr{round( exp(-(VQu+VQv)/2/VQl-(VQu-VQv)/2/VQll)/VQl/VQll/2 ,10)}
  Soient $X$ et $Y$ deux variables aléatoires indépendantes de lois
  exponentielles de paramètres respectifs $\lambda=1/\VQl$ et
  $\lambda'=1/\VQll$. On note $g$ la densité du vecteur $(X+Y,X-Y)$.

  À quel intervalle appartient $g(\VQu,\VQv)$ ? \indique{\VQr}

  \begin{multicols}{5}
    \begin{reponses}[o]
      \choixIntervalles{\VQr}{0}{0.1}{0.005}
    \end{reponses}
  \end{multicols}
\end{question}
}


%Q[Calc-SOA109]
\element{grq}{
\begin{question}{Calc-SOA109}\bareme{b=3}
  \FPeval\VQc{trunc(2+random*4,0)}

  Soient $X$ et $Y$ deux variables aléatoires indépendantes de lois
  respectives $\Exp(1)$ et $\Exp(1/\VQc)$ (lois exponentielles).  On
  note $R=X/Y$. Pour $x>0$, quelle est la valeur en $x$ de la
  densité de $R$ ?

  \begin{multicols}{2}
    \begin{reponses}
      \mauvaise{$\displaystyle \frac1{\VQc x+1}$}
      \bonne{$\displaystyle \frac{\VQc}{(\VQc x+1)^2}$}
      \mauvaise{$e^{-x}$}
      \mauvaise{$\VQc e^{-\VQc x}$}
      \mauvaise{$x e^{-x}$}
    \end{reponses}
  \end{multicols}
\end{question}
}

%Q[Calc-Qx-expsym]
\element{grq}{
\begin{question}{Calc-Qx-expsym}\bareme{b=3}
  \FPeval\VQl{trunc(2*trunc(1+2*random,0)+1,0)}
  \FPeval\VQp{trunc((6+4*random)/10,1)}
  \FPeval\VQr{round(neg(ln(2-2*VQp))/VQl,10)}
  Soit $X$ une variable aléatoire de densité
  \[ f(x)= \frac{\VQl}2\exp(-\VQl|x|)\,,\]
  et de fonction quantile $Q$. À quel intervalle appartient $Q(\VQp)$ ?
  \begin{multicols}{5}
    \begin{reponses}[o]
      \choixIntervalles{\VQr}{-1}{1}{0.1}
    \end{reponses}
  \end{multicols}
\end{question}
}

%Q[Calc-SOA-frep]
\element{grq}{
\begin{question}{Calc-SOA-frep}\bareme{b=3}
  \FPeval\VQa{trunc(2+2*random,0)}
  \FPeval\VQb{trunc(2+2*random,0)}
  \FPeval\VQc{trunc(2+2*random,0)}
  \FPeval\VQr{round( (((VQa+1)/VQa)^(1-VQb)-1)/(((VQa+VQc)/VQa)^(1-VQb)-1),10)}
  Soit $X$ une variable aléatoire de densité
  \[ f(x)=k(\VQa+x)^{-\VQb}\II_{[0,\VQc]}(x)\,,\] où $k$ est une
  constante réelle. À quel intervalle appartient la probabilité
  $\P{X<1}$ ?
  
  \choixProba{\VQr}
\end{question}
}

%Q[Calc-SOA45]
\element{grq}{
\begin{question}{Calc-SOA45}\bareme{b=3}
  \exemplairepair
  \FPeval\a{trunc(1 + random * 2,0)}
  \FPeval\b{trunc(3 + random * 2,0)}
  \else
  \FPeval\a{trunc(3 + random * 2,0)}
  \FPeval\b{trunc(1 + random * 2,0)}
  \fi
  \FPeval\r{clip( 2*(b*b*b-a*a*a)/(3*(a*a+b*b)) )}
  Soit $X$ une variable aléatoire de densité
  \[ f(x)=k\,|x|\,\II_{[-\a,\b]}(x)\,,    \]
  où $k$ est une constante réelle.
  À quel intervalle appartient l'espérance de $X$ ? \indique{\r}
  \begin{multicols}{5}
    \begin{reponses}[o]
      \choixIntervalles{\r}{-3}{3}{0.5}
    \end{reponses}
  \end{multicols}
\end{question}
}

%Q[Calc-SOA77]
\element{grq}{
\begin{question}{Calc-SOA77}\bareme{b=3}
  \FPeval\VQa{trunc(2+3*random,0)}
  \FPeval\VQd{clip(12*VQa+16)}
  \FPeval\VQr{clip(round( (18+15*VQa)/(2*VQd) ,10))}

  Soit $(X,Y)$ un vecteur aléatoire continu de densité
  \[ f(x,y)=\left\{
    \begin{array}{ll}
      \displaystyle\frac{3(\VQa x+y^2)}{\VQd}& \mbox{si $0<x<2$ et $0<y<2$} \\
      0 & \mbox{sinon}
    \end{array}\right.
  \]
  À quel intervalle appartient la probabilité $\P{\min(X,Y)<1}$ ? \indique{\VQr}
  
  \choixProba{\VQr}
\end{question}
}

%Q[Calc-var-unif-ab]
\element{grq}{
\begin{question}{Calc-var-unif-ab}\bareme{b=2}
  \FPeval\VQa{clip(0-trunc(1+random*3,0))}
  \FPeval\VQb{trunc(10+random*3,0)}
  \FPeval\VQr{clip(round( (VQb-VQa)*(VQb-VQa)/12 ,10))}

  À quel intervalle appartient la variance de la loi uniforme sur
  $[\VQa,\ \VQb]$ ? \indique{\VQr}

  \begin{multicols}{5}
    \begin{reponses}[o]
      \choixIntervalles{\VQr}{8}{25}{1}
    \end{reponses}
  \end{multicols}
\end{question}    
}


%%%%%%%%%%%%%%%%%%%%%%%%%%%%%%%%%%%%%%%%%%%%%%%%%%%%%%%%%%%%%%%%%%%%%%
%%%%%%%%%%%%%%%%%%%%%%%%%%%%%%%%%%%%%%%%%%%%%%%%%%%%%%%%%%%%%%%%%%%%%%

\exemplaire{5}{
\noindent{\bf ISFA  \hfill Probabilités -- Examen du 06/12/2010 \hfill 1\iere{} année}

\begin{center}\em
Durée : 1h20.

  Aucun document n'est autorisé.
  L'usage d'une {\bf calculatrice non programmable} est autorisé.

  Les questions ont toutes une unique bonne réponse.
\end{center}
\vspace{1ex}


{\setlength{\parindent}{0pt}\hspace*{\fill}\AMCcode{etu}{8}\hspace*{\fill}
\begin{minipage}{5.5cm}
$\longleftarrow{}$\hspace{0pt plus 1cm} codez votre numéro d'étu\-diant ci-contre, et écrivez votre nom et prénom ci-dessous.

\vspace{3ex}

\champnom{\fbox{    
    \begin{minipage}{.86\linewidth}
      Nom et prénom :
      
      \vspace*{.5cm}\dotfill
      
      \vspace*{.5cm}\dotfill
      \vspace*{1mm}
    \end{minipage}
  }}\end{minipage}\hspace*{\fill}
}

\vspace{1ex}

\noindent\hrulefill

\vspace{2ex}

%%%%%%%%%%%%%%%%%%%%%%%%%%%%%%%%%%%%%%%%%%%%%%%%%%%%%%%%%%%%%%%%%%%%%%
%%%%%%%%%%%%%%%%%%%%%%%%%%%%%%%%%%%%%%%%%%%%%%%%%%%%%%%%%%%%%%%%%%%%%%

\melangegroupe{grq}

\restituegroupe{grq}


\cleardoublepage %
}

\end{document}
